% Created 2018-12-06
\documentclass[a4paper]{article}
\usepackage[utf8]{inputenc}
\usepackage[T1]{fontenc}
 \usepackage{lmodern}
\usepackage{fixltx2e}
\usepackage{graphicx}
\usepackage{longtable}
\usepackage{float}
\usepackage{wrapfig}
\usepackage{rotating}
\usepackage[normalem]{ulem}
\usepackage{amsmath}
\usepackage{textcomp}
\usepackage{marvosym}
\usepackage{wasysym}
\usepackage{amssymb}
\usepackage{hyperref}
\hypersetup{
colorlinks=true,
linkcolor=blue,
pdfauthor=Germain Salgues,
pdftitle=Algorithm theoretical basis documentation for the cold cloud removal in
the snow cover product (Let-it-snow)}

\tolerance=1000
\usepackage{amsfonts,bm}
\usepackage{color}
\usepackage[usenames,dvipsnames]{xcolor}
\usepackage[margin=2.5cm,a4paper]{geometry}
\usepackage{enumitem}
\usepackage{fancyhdr}
\usepackage{tabularx}

\usepackage{algorithm}
\usepackage[noend]{algpseudocode}

\makeatletter
\def\BState{\State\hskip-\ALG@thistlm}
\makeatother

\renewcommand{\maketitle}{}
\date{\today}
% \title{ATBD CES surface enneigée}
% \hypersetup{
%   pdfkeywords={},
%   pdfsubject={},
%   pdfcreator={Emacs 24.3.1 (Org mode 8.2.4)}}
\begin{document}

\maketitle
\pagestyle{fancy}
% \providecommand{\alert}[1]{\textbf{#1}}
% \setlist[itemize,1]{label=$\diamond$}
% \setlist[itemize,2]{label=$\ast$}
% \setlist[itemize,3]{label=$\star$}
% \setlist[itemize,4]{label=$\bullet$}
% \setlist[itemize,5]{label=$\circ$}
% \setlist[itemize,6]{label=$-$}
% \setlist[itemize,7]{label=$\cdot$}
% \setlist[itemize,8]{label=$\cdot$}
% \setlist[itemize,9]{label=$\cdot$}
% \renewlist{itemize}{itemize}{9}
\lhead[]{\includegraphics[width=0.1\textwidth]{./images/logo_cesbio.png}}
\rhead[]{\thepage}
% \cfoot{\textcolor{PineGreen}{copyright?}}

\begin{titlepage}
\includegraphics[width=0.3\textwidth]{./images/logo_cesbio.png}
\hspace{5cm}
\includegraphics[width=0.3\textwidth]{./images/Theia_en.png}

\vspace{3cm}



\textcolor{PineGreen}{ \huge \bfseries Theia Land Data Centre\\ }
% \vspace{0.5cm}
\rule{\linewidth}{0.5mm}
\begin{center}
{ \huge \bfseries Algorithm theoretical basis documentation for the cold cloud removal in
the snow cover product (Let-it-snow)\\}
\rule{\linewidth}{0.5mm}
{ \large \bfseries Simon Gascoin (CNRS/CESBIO), Manuel Grizonnet (CNES), Germain Salgues (Magellium) \\ }
{ \large \bfseries V1.0 (Updated for LIS 1.5) - \today \\ }

% \vspace{3cm}
% \includegraphics[width=1\textwidth]{./Images/Theia_en.png}


\end{center}
\end{titlepage}


\section{Description}\label{par:desc}

This document describes the algorithm of this intermediate processing related to
the special case where high cold clouds are detected as snow area during the
first pass of the snow detection algorithm. The assumption is that only the
central area of the high cold cloud is flagged as snow because of their lower
temperature. In consequences, these miss detected areas will always be
surrounded by clouds.

The following algorithm aims at detecting snow areas that are totally or
partially surrounded by clouds and correcting them.

Please note that in few cases snow area surrounded by clouds may not be a miss
classification. For example, some mountain summits may actually be snow covered
while lower clouds surround them.


\section{Algorithm}\label{par:algo}

\begin{algorithm}
\caption{Cold Cloud removal}\label{cold_cloud}
\begin{algorithmic}[1]
\Procedure{pass1.5(snow\_polygons)}{}
\If {\textit{rm\_snow\_inside\_cloud} is True}
\State \For{\textbf{each} $snow\_polygon$ in $snow\_polygons$}
\State \If {$area(snow\_polygon) > rm\_snow\_inside\_cloud\_min\_area$}
\State \State $dilatation(snow\_polygon,rm\_snow\_inside\_cloud\_dilation\_radius)$
\State \State \Comment{A dilation of rm\_snow\_inside\_cloud\_dilation\_radius pixels is
  applied onto the current snow area.}
\State \State \Comment{The crown surrounding the current snow area is obtained by the subtraction of the newly dilated snow area and the initial snow area.}
\State \State \Comment{The crown pixel are compared to the cloud mask obtain during the
first pass, to obtain the percentage of cloud pixels surrounding the current
snow area ($cloud\_percent$).}
\State \State \If {$cloud\_percent > rm\_snow\_inside\_cloud\_threshold$}
\State \State \State $snow\_polygon = 0$ \Comment{The snow mask is updated to remove}
the current snow area
\State \State \State $cloud\_mask=1$ \Comment{The cloud mask is updated to add the current snow area as a new cloud area}
\State \State \EndIf
\State \EndIf
\State \EndFor
\State \Comment{The updated masks (both snow and cloud) are then used as inputs
  for the next snow detection stet (pass 2).} 
\EndIf
\EndProcedure
\end{algorithmic}
\end{algorithm}

\end{document}
